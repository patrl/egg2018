\documentclass[nofonts,nobib]{tufte-handout}

\title{MaxElide}

\ifxetex
  \renewcommand{\textls}[2][5]{%
    \begingroup\addfontfeatures{LetterSpace=#1}#2\endgroup
  }
  \renewcommand{\allcapsspacing}[1]{\textls[15]{#1}}
  \renewcommand{\smallcapsspacing}[1]{\textls[10]{#1}}
  \renewcommand{\allcaps}[1]{\textls[15]{\MakeTextUppercase{#1}}}
  \renewcommand{\smallcaps}[1]{\smallcapsspacing{\scshape\MakeTextLowercase{#1}}}
  \renewcommand{\textsc}[1]{\smallcapsspacing{\textsmallcaps{#1}}}
\fi
% trees
\PassOptionsToPackage{linguistics}{forest}
\PassOptionsToPackage{nolist,nohyperlinks}{acronym}
\PassOptionsToPackage{normalem}{ulem}

\usepackage{
  % needed for towers
  booktabs,
  forest,
  expex,
  acronym,
  todonotes,
  multicol,
  braket,
  amsmath,
  stmaryrd,
  pifont,
  ulem
}

\usepackage{fontspec}

\setmainfont{STIX2Text}[
Path = ../fonts/,
Extension       = .otf,
UprightFont     = *-Regular,
ItalicFont      = *-Italic,
BoldFont        = *-Bold,
BoldItalicFont  = *-BoldItalic ]

\usepackage[math-style=ISO]{unicode-math}
\setmathfont{STIX2Math}[
Path = ../fonts/,
Extension = .otf
]

\setsansfont[Scale=MatchLowercase]{TeX Gyre Heros}

\setmonofont{Input Mono}

\usepackage[tracking=false]{microtype}

\PassOptionsToPackage{backend=biber,bibstyle=biblatex-sp-unified,citestyle=sp-authoryear-comp,url=false,doi=false,bibencoding=utf8}{biblatex}
\usepackage{biblatex}


\addbibresource[location=remote]{/home/patrl/GitHub/bibliography/elliott_mybib.bib}
\newcommand{\eval}[2][]{\ensuremath{⟦ \text{#2} ⟧^{#1}}}
\newcommand{\evalM}[2][]{\ensuremath{⟦ #2 ⟧^{#1}}}
\newcommand{\citepossessive}[1]{\citeauthor{#1}'s (\citeyear{#1})}
\newcommand{\citepossessivealt}[1]{\citeauthor{#1}'s \citeyear{#1}}
\newcommand{\sub}[1]{\textsubscript{#1}}
\newcommand{\supscr}[1]{\textsuperscript{#1}}
\newcommand{\lam}[1]{\ensuremath{λ #1\,.\,}}
\newcommand{\type}[1]{\ensuremath{\mathsf{#1}}}
\newcommand{\metalang}[1]{\ensuremath{\textsf{#1}}}
\newcommand{\thFunc}[1]{\ensuremath{\text{\textsc{#1}}}}
\newcommand{\agent}[2][]{\ensuremath{\text{\textsc{ag}}_{#1}(#2)}}
\newcommand{\agentNoArg}{\ensuremath{\text{\textsc{ag}}}}
\newcommand{\theme}[2][]{\ensuremath{\text{\textsc{th}}_{#1}(#2)}}
\newcommand{\themeNoArg}{\ensuremath{\text{\textsc{th}}}}
\newcommand{\cont}[2][]{\ensuremath{\text{\textsc{cont}}_{#1}(#2)}}
\newcommand{\contNoArg}{\ensuremath{\text{\textsc{cont}}}}
\newcommand{\holder}[2][]{\ensuremath{\text{\textsc{holder}}_{#1}(#2)}}
\newcommand{\holderNoArg}{\ensuremath{\text{\textsc{holder}}}}
\newcommand{\dox}[2]{\ensuremath{\mathrm{Dox}_{#1,#2}}}
\newcommand{\sayAlt}[2]{\ensuremath{\mathrm{Say}_{#1,#2}}}
\newcommand{\dmroot}[1]{\ensuremath{\sqrt{\text{#1}}}}
\newcommand{\cmark}{\ding{51}}%
\newcommand{\xmark}{\ding{55}}%
\newcommand{\aktHead}{AKT\textsuperscript{0}}
\newcommand{\senClo}{\ensuremath{\exists_{\mathrm{SEN}}}}
\newcommand{\rnClo}{\ensuremath{\exists_{\mathrm{RN}}}}
\newcommand{\intHead}{F\sub{int}}
\newcommand{\extHead}{F\sub{ext}}
\newcommand{\pluralOp}{\ensuremath{*}}
\newcommand{\eventClo}{\ensuremath{𝓔}}
\newcommand{\eventCloPl}{\ensuremath{*∃}}
\newcommand{\trace}[1][]{\textit{t\textsubscript{#1}}}
\newcommand{\bracketStr}[2][]{[\textsubscript{#1}\,{#2}\,]}
\newcommand{\liftOp}{\ensuremath{⇑}}
\newcommand{\powerset}[1]{\ensuremath{𝒫(#1)}}
\newcommand{\perfective}{\textsc{perf}}
\newcommand{\imperf}{\textsc{imperf}}
\newcommand{\contOp}{\ensuremath{𝕮}}
\newcommand{\boolMeet}{\ensuremath{∧}}
\newcommand{\boolJoin}{\ensuremath{∨}}
\newcommand{\boolCompNoArg}{\ensuremath{\overline{\phantom{x}}}}
\newcommand{\boolComp}[1]{\ensuremath{\overline{#1}}}
\newcommand{\boolDom}{\ensuremath{≤}}
\newcommand{\maxSort}{\ensuremath{\text{\textsc max}}}
\newcommand{\boolType}{\ensuremath{\mathtt{type_B}}}
\newcommand{\extType}{\ensuremath{\mathtt{type}}}
\newcommand{\primTypes}{\ensuremath{\mathtt{type_0}}}
\newcommand{\boolTypeVar}{\ensuremath{τ}}
\newcommand{\typeVar}{\ensuremath{σ}}
\newcommand{\doxStates}{\ensuremath{\text{\textsc doxStates}}}
\newcommand{\strawsonEntails}{\ensuremath{⊧_S}}
\newcommand{\nstrawsonEntails}{\ensuremath{̸⊧_S}}
\newcommand{\altInq}[1]{\ensuremath{\mathrm{alt}_{\mathrm{Inq}}(#1)}}
\newcommand{\ansOp}{\ensuremath{\text{\textsc{ans}\sub{Q}}}}
\newcommand{\alt}[2][]{\ensuremath{\text{\textsc{alt}}{#1}(#2)}}
\newcommand{\excl}[1]{\ensuremath{\text{\textsc{excl}}^{\text{\textsc{ie}}}(#1)}}
\newcommand{\domain}[1]{\ensuremath{\mathsf{dom}}}
\newcommand{\elide}[1]{\textcolor{gray}{\dashuline{#1}}}

% % requires the booktabs package
\newcommand\semtower[2]{% a 2-level semantic tower
  \begin{tabular}[c]{@{\,}c@{\,}}
    \(#1\)
    \\
    \midrule
    \(#2\)
    \\
  \end{tabular}
}
\newcommand\tower[3]{% a 2-level type/category tower
  \begin{tabular}[c]{@{\,}c@{\,}}
    \(\hfil #1\ \vrule width .05em\ #2 \hfil\)
    \\
    \midrule
    \(#3\)
    \\
  \end{tabular}
}


\setcounter{secnumdepth}{3}

\setlength{\parindent}{0pt}

\author{Patrick D. Elliott}

\date{\today}

\begin{document}

\begin{acronym}
\acro{MP!}{Maximize Presupposition!}%
\end{acronym}

\maketitle

\section{Strict vs.\,sloppy}

\begin{itemize}

  \item We've already seen that elliptical constructions involving bound variables can give rise to an ambiguity.
    
    \ex
    Ivan sold his telecaster. Jorge also did Δ.
    \xe

    \pex~
    \a Δ = sell Ivan's telecaster.\marginnote{strict}  
    \a Δ = sell Jorge's telecaster.\marginnote{sloppy}
    \xe
    
  \item On an approach to ellipsis assuming \emph{silent structure}, the classical
   analysis of the sloppy reading is to posit a bound variable inside of the
   EC.
   
    \ex
    Ivan\(^{x}\) [\(_{\text{VP}}\) sold his\(_{x}\) telecaster].\newline
    Jorge\(^{y}\) also did \elide{sell his\(_{y}\) telecaster}. 
    \xe

  \item Question: how is the \emph{identity condition} on ellipsis met under the sloppy reading?
    
    \ex
    \label{parallelism}\emph{The parallelism condition}\newline
    Ellipsis of EC requires semantic identity with with an antecedent constituent AC.
    \xe


  \item The pronoun in the AC picks out \emph{Ivan}, but the pronoun in the EC picks out \emph{Jorge}.\sidenote{The functions in (\ref{clauses}) aren't \(α\)-equivalent, since there are contexts which map $x$ and $y$ to distinct individuals.}

    \pex\label{clauses}
    \a \(\eval[g]{AC} = x\,\metalang{sold}\,\metalang{telecaster}(x)\)
    \a \(\eval[g]{EC} = y\,\metalang{sold}\,\metalang{telecaster}(y)\)
    \xe
     
  \item \citet{sag1976} and \citet{williams_discourse_1977} argue that, under the sloppy reading, the AC and the EC \emph{are} in fact identical - the pronoun is interpreted as a variable bound by a VP-internal λ-operator.\sidenote{
    Note that, since each occurrence of the variable $y$ is bound by a matching \(λ\)-operator, the choice of variable name in the EC doesn't matter here for the purposes of semantic identity. In other words, VP-A and VP-E denote \(α\)-equivalent functions.
    }

    \ex   
    Ivan [\( λ x .  x \) sold \(x\)'s telecaster].\newline
    Jorge also did \elide{\( λ y . y\) sold \(y\)'s telecaster}.
    \xe
    
    
\end{itemize}

\section{Re-binding}

\subsection{Re-binding vs.\,internal binding}

\begin{itemize}

  \item The Sag-Williams account of the strict-sloppy ambiguity makes a straightforward prediction \emph{internal binding} as in (\ref{intbinding}), is allowed, but \emph{re-binding} as in (\ref{rebinding}), is not. 
    
    \begin{multicols}{2}

    \pex\label{rebinding}\emph{Re-binding} \xmark
    \a\label{rebindingA}[ \ldots\,\(λ x\) \ldots\,[\(_{\text{AC}}\) \ldots\,\(x\) \ldots\,]]
    \a\label{rebindingE}[ \ldots\,\(λ y\) \ldots\,\elide{[\(_{\text{EC}}\) \ldots\,\(y\) \ldots\,]}]
    \xe
    
    \columnbreak
    
    \pex\label{intbinding}\emph{Internal binding} \cmark
    \a\label{intbindingA}[ \ldots\,[\(_{\text{AC}}\) \ldots\,\(λ\,x\)\ldots\,\(x\) \ldots\,]]
    \a\label{intbindingE}[ \ldots\,\elide{[\(_{\text{EC}}\) \ldots\,\(λ\,y\)\,\ldots\,\(x\) \ldots\,]}]
    \xe
    
    \end{multicols}
    
  \item In other words, \emph{sloppy identity} requires an EC-internal binder, and a corresponding AC-internal binder.

\end{itemize}

\subsection{Evidence for the Sag-Williams position}

\begin{itemize}

  \item Internal binding should only be possible if the understood antecedent of the variable is the sister of the EC. \emph{Parallelism} therefore rules out (\ref{rebinding2b}).

    \pex\label{rebinding2}
    \a\label{rebinding2a}Jorge said that Tanya likes him\(_{J}\),\newline
    and \textsc{Ivan} also did \elide{\( λ x . x \) say that Tanya likes \(x\)}.
    \a\ljudge{*}\label{rebinding2b}Jorge said that Tanya likes him\(_{J}\),\newline
    and \textsc{Ivan} also \( λ x . x \) said that she does \elide{\( λ y . y \) like \(x\)}.
    \xe
    
  \item Consider also the following example from \citet[p.\,225]{fox_maxelide_2005}, where wh-movement creates the re-binding configuration:\sidenote{Note that, because of the possibility of \emph{successive-cycle wh-movement}, it is not immediately obvious that the parallelism condition rules out the configuration in (\ref{foxtak1}).

    Concretely, why can't wh-movement leave behind an intermediate trace adjacent to the EC?
    
    \ex
    which one \( λ x .  \) we did \(t_{x}\) \elide{\( λ x . λ y . y\) invite \(x\)}
    \xe

}
    
    \pex\label{foxtak1}
    \a\label{foxtak1a}John knows which professor we invited,\newline
    but he is not allowed to reveal which one \elide{\( λ x . x \) we invited \(x\)}.
    \a\ljudge{*}\label{foktak1b}John knows which professor we invited,\newline
    but he is not allowed to reveal\newline
    which one \( λ x .  \) we did \elide{\( λ y . y \) invite \(x\)}.
    \xe

\end{itemize}

\subsection{Evidence against the Sag-Williams position}

\begin{itemize}

  \item Sometimes, it looks like sloppy identity is possible even where internal binding is not:

    \ex
    \label{evAgainst}Jorge insists that Tanja likes him,\newline
    but \textsc{Ivan} \( λ x . x \) \uwave{\textsc{denies}} that she does \elide{\( λ y .  y\) like \(x\)}
    \xe
    
  \item The same point goes for rebinding configurations created by movement (\citealt[p\,226]{fox_maxelide_2005}):

\ex
\label{evAgainst2}Mary doesn't know who we can invite,\newline
    but she can tell you who \( λ x .  \) we can \uwave{\textsc{not}} \elide{\( λ y . y \) invite \(x\)}
\xe
    
  \item The empirical generalization that \citet{fox_maxelide_2005} suggest is the following:

    \ex
    \label{interv}\emph{Intervening focus:}\newline
    when a focused-expression intervenes between the re-binder and and the re-bound variable, ellipsis is licensed.
    \xe
    
\end{itemize}
    
\section{Enter \textsc{MaxElide}}

\begin{itemize}

  \item \todo[inline]{Add maxelide references here}
    
\ex \textsc{MaxElide} (alpha version)\newline
Elide the largest deletable constituent.
\xe
    
  \item In all of our unacceptable cases, there was a bigger constituent that \emph{could've} been deleted.\sidenote{VP-ellipsis conditions the spell-out of T, which distracts from the parallel between (\ref{rebinding3a}) and (\ref{rebinding3b}). I've used \emph{do}-insertion here to draw attention to the parallel, even though it sounds a little unnatural.}
    
\pex\label{rebinding3}Jorge said that Tanya likes him\(_{J}\)
    \a\label{rebinding3a}\ldots and \textsc{Ivan} also did \elide{\( λ x . x \) say that she does \( λ y . y \) like \(x\)}.
    \a\ljudge{*}\label{rebinding3b}\ldots and \textsc{Ivan} also did \( λ x . x \) say that she does \elide{\( λ y . y \) like \(x\)}.
    \xe
    
  \item In the acceptable cases, however, the presence of an intervening focus blocks deletion of the larger constituent.
    
  \item (\ref{evAgainst2a}) isn't a candidate deletion, since it violates the parallelism condition.
    
\pex
    \label{evAgainst2}Jorge insists that Tanja likes him,
    \a\label{evAgainst2a}\ljudge{\xmark}\ldots but \textsc{Ivan} \elide{\( λ x . x \) \textsc{denies} that she does \elide{\( λ y .  y\) like \(x\)}}
    \a\label{evAgainst2b}\ljudge{\cmark}\ldots but \textsc{Ivan} \( λ x . x \) \uwave{\textsc{denies}} that she does \elide{\( λ y .  y\) like \(x\)}
    \xe

\end{itemize}

\subsection{Circumvention of \textsc{MaxElide}}

\begin{itemize}

  \item It's not hard to come up with counter-examples to \textsc{MaxElide}:

    \pex Jorge said that Tanya likes Ivan,
    \a \ldots \textsc{Robert} also said she does \elide{like Ivan}.
    \a \ldots \textsc{Robert} also did \elide{say that she likes Ivan}.
    \xe
    
  \item \todo[inline]{Add Merchant reference}
    
    \ex\label{maxElideBeta}
    \textsc{MaxElide} (beta version)\newline
    Elide the biggest deletable constituent if EC contains a variable that is free within EC.
    \xe
    
  \item The constraint in (\ref{maxElideBeta}) is still not quite general enough. \citet{fox_maxelide_2005} observe that it fails to rule out \emph{co-binding} configurations:
    
    \pex\label{cobinding}I know which puppy \( λ x .  \) \ldots
    \a\label{cobinding1}\ldots you said Mary would adopt\newline
    and Fred did \elide{say she would adopt \(x\)} too.
    \a\label{cobinding2}\ldots you said Mary would adopt\newline
    and Fred said she would \elide{adopt \(x\)} too.
    \xe
    
  \item In \emph{co-binding}, unlike \emph{re-binding} configurations the variables free within AC and EC are re-bound by the same re-binder.
    
  \item \citet{fox_maxelide_2005} suggest the following refinement of \textsc{MaxElide}:

    \ex
    \label{final}\textsc{MaxElide} \(1.0\)\newline
    Elide the biggest deletable constituent in a re-binding configuration.\newline
    A re-binding configuration is a structure in which EC dominates a variable that is free within EC and is bound by a binder YP outside of EC, and there is no variable in AC also bound by YP.
    \xe
    
  \item This seems to get the right empirical coverage, but is it plausible that such a complicated principle is codified in the grammar? Can it be reduced to more basic principles?
    
\end{itemize}
    
\section{\citeauthor{fox_maxelide_2005}'s (\citeyear{fox_maxelide_2005}) proposal}
    
\begin{itemize}
        
  \item \citet{fox_maxelide_2005} propose that, in order to understand where \textsc{MaxElide} applies, we must expand the domain in which the parallelism condition applies beyond just what is deleted.
        
  \item Instead, \citeauthor{fox_maxelide_2005} adopt a revised parallelism condition based on \citet{Rooth}.
    
    \ex
    For ellipsis of EC to be licensed, there must exist a \emph{Parallelism Domain} (PD). 
    \xe
    
    \pex \emph{Parallelism Domain:}\newline
    An XP is a PD of EC iff:
    \a XP reflexively dominates EC.
    \a There exists a constituent \(ϕ \in \metalang{alt}(\text{XP})\), s.t. for all assignments \(g\), \(\eval[g]{ϕ} = \eval[g]{AC}\).
    \xe
    
  \item \citeauthor{fox_maxelide_2005} propose that \textsc{MaxElide} is relativized to the PD:

    \ex
    \label{maxelideF}\textsc{MaxElide} \(1.1\)\newline
    Elide the biggest deletable constituent reflexively dominated by a PD.
    \xe
    
  \item Let's see how this accounts for \textsc{MaxElide} circumvention.
    
    \pex Jorge said that Tanya likes Ivan,
    \a \ldots \textsc{Robert} also said she does \elide{like Ivan}.
    \a \ldots \textsc{Robert} also did \elide{say that she likes Ivan}.
    \xe
    
  \item In the above example, both possible ellipses are PDs, and therefore \textsc{MaxElide} is trivially satisfied.
    
  \item In re-binding configurations however, the PD must at least contain the binder, and \textsc{MaxElide} \(1.1\) demands deletion of the biggest constituent within the PD.

    \pex\label{rebindingRep3}Jorge said that Tanya likes him\(_{J}\)
    \a\label{rebindingRep3a}\ldots and \textsc{Ivan} also did [\(_{\text{PD}}\) \elide{\( λ x . x \) say that she does \( λ y . y \) like \(x\)}].
    \a\ljudge{*}\label{rebindingRep3b}\ldots and \textsc{Ivan} also did [\(_{\text{PD}}\) \( λ x . x \) say that she does \elide{\( λ y . y \) like \(x\)}].
    \xe       

  \item Just as before, intervening focus shrinks the candidates for deletion:

    \pex
    \label{evAgainstRep2}Jorge insists that Tanja likes him,
    \a\label{evAgainstRep2a}\ljudge{\xmark}\ldots but \textsc{Ivan} [\(_{\text{PD}}\) \elide{\( λ x . x \) \textsc{denies} that she does \( λ y .  y\) like \(x\)}]
    \a\label{evAgainstRep2b}\ljudge{\cmark}\ldots but \textsc{Ivan} [\(_{\text{PD}}\) \( λ x . x \) \uwave{\textsc{denies}} that she does \elide{\( λ y .  y\) like \(x\)}
    \xe
    
  \item Things get a little more interesting with \emph{co-binding} configurations.
    
    \pex\label{cobindingRep}I know which puppy \( λ x .  \) \ldots
    \a\label{cobindingRep1}\ldots you said Mary would adopt \(x\)\newline
    and Fred did [\(_{\text{PD}}\) \elide{say she would adopt \(x\)}] too.
    \a\label{cobindingRep2}\ldots you said Mary would adopt \(x\)\newline
    and Fred said she would [\(_{\text{PD}}\) \elide{adopt \(x\)}] too.
    \xe
    
  \item \citeauthor{fox_maxelide_2005}'s idea is that, since in (\ref{cobindingRep2}), the variables in AC and EC are bound by the same binder, they share the same variable name, and therefore the embedded VP is also a PD. 
    
  \item This is trivially the case -- \(\evalM[g]{\text{adopt}\,x} = \evalM[g]{\text{adopt}\,x}\) for all assignments.

  \item In co-binding configurations, circumvention of \textsc{MaxElide} crucially depends on identity of variable names.
    
  \item This begs the question: what prevents us from \emph{always} circumventing MaxElide by simply always picking identical variable names, even when the re-binders are distinct?, i.e.
    
    \pex
    \a\label{circ}[ \ldots\,\(λ x\) \ldots [\(_{\text{AC}}\) \ldots\,\(x\) \ldots\,]]
    \a\label{circ2}[ \ldots\,\(λ x\) \ldots \elide{[\(_{\text{EC}}\) \ldots\,\(x\) \ldots\,]}]
    \xe
    
  \item To block this possibility, \citeauthor{fox_maxelide_2005} make recourse to Heim's \emph{no meaningless co-indexation} constraint. \todo[inline]{Add proper Heim citation}
    
\ex
    \label{meaningless}\emph{No meaningless co-indexation}\newline
    If an LF contains an occurrence of a variable \(v\) that is bound by a node \(α\), then all occurrences of \(v\) in this LF must be bound by the same node as \(α\).
\xe
    
\end{itemize}
    
\section{Re-binding and covert movement}
    
\begin{itemize}

  \item It is well known that QR can create a re-binding configuration.
    
  \item First off, observe that a quantificational object can take wide scope out of the EC.

    \ex
    \label{qr}A doctor treated every patient.\newline
    A \textsc{nurse} did \elide{treat every patient} too.\marginnote{\(∃ > ∀; ∀ > ∃\)}
    \xe
    
    \pex~\label{qr2}
    \a\label{qr2a} AC: every patient \( λ x \) a doctor treated \(x\)
    \a\label{qr2b} EC: every patient [\(_{\text{PD}}\) \(λ\,y\) \textsc{a nurse} \elide{treated \(y\)}]
    \xe
    
  \item Since \emph{a nurse} is focused, the VP is indeed the biggest deletable constituent in the PD.
    
  \item If covert movement can create a re-binding configuration, we expect that there should be circumstances under which sloppy identity is ruled out. 
    
    \pex
    At least one doctor tried to get me to arrest every patient,
    \a \ldots and at least one \textsc{nurse} tried to get me to \elide{arrest every patient} as well.\marginnote{\(∃ > ∀; \text{*} ∀ > ∃\)}
    \a \ldots and at least one \textsc{nurse} did \elide{try to get me to arrest every patient} as well.\marginnote{\(∃ > ∀; ? ∀ > ∃\)}
    \xe
    
\end{itemize}
    
\section{Re-binding and scope}

\begin{itemize}
    
  \item question here: is re-binding about \emph{covert movement}, or \emph{scope}?
    
\ex
Each brother hopes that old age will kill a certain relative of theirs soon, and each sister hopes that it will too.
\xe
    
\ex~
Each brother hopes that old age will kill a certain relative of theirs soon, and each sister does too.
\xe
    
\section{Donkey re-binding}

\ex
Every farmer who owns a donkey\(^{x}\) asked Mary to beat it\(_{x}\),\newline
and every \textsc{landlord} who owns a donkey did too. 
\xe
    
\ex
Every farmer who owns a donkey\(^{x}\) asked Mary to beat it\(_{x}\),\newline
and every \textsc{landlord} who owns a donkey asked her to as well. 
\xe
    

    
\end{itemize}

\newpage

\begin{fullwidth}
\printbibliography
\end{fullwidth}

\end{document}

%%% Local Variables:
%%% mode: latex
%%% TeX-engine: xetex
%%% TeX-master: t
%%% End:
